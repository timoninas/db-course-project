\documentclass[12pt, a4paper]{report}
\usepackage[utf8]{inputenc}
\usepackage[english, russian]{babel}

\usepackage{graphicx}
\usepackage{listings}
\usepackage{color}

\usepackage{amsmath}
\usepackage{pgfplots}
\usepackage{url}
\usepackage{flowchart}
\usepackage{tikz}
\DeclareGraphicsExtensions{.pdf,.png,.jpg,.svg}
\usetikzlibrary{shapes, arrows}

\usepackage{pgfplotstable}

\renewcommand\contentsname{Содержание}

\usepackage{geometry}
\geometry{left=3cm}
\geometry{right=1cm}
\geometry{top=2cm}
\geometry{bottom=2cm}

\lstset{ %
language=Swift,                 % выбор языка для подсветки (здесь это С)
basicstyle=\small\sffamily, % размер и начертание шрифта для подсветки кода
numbers=left,               % где поставить нумерацию строк (слева\справа)
numberstyle=\tiny,           % размер шрифта для номеров строк
stepnumber=1,                   % размер шага между двумя номерами строк
numbersep=-5pt,                % как далеко отстоят номера строк от         подсвечиваемого кода
backgroundcolor=\color{white}, % цвет фона подсветки - используем         \usepackage{color}
showspaces=false,            % показывать или нет пробелы специальными     отступами
showstringspaces=false,      % показывать или нет пробелы в строках
showtabs=false,             % показывать или нет табуляцию в строках
frame=single,              % рисовать рамку вокруг кода
tabsize=2,                 % размер табуляции по умолчанию равен 2 пробелам
captionpos=t,              % позиция заголовка вверху [t] или внизу [b] 
breaklines=true,           % автоматически переносить строки (да\нет)
breakatwhitespace=false, % переносить строки только если есть пробел
escapeinside={\%*}{*)},   % если нужно добавить комментарии в коде
keywordstyle=\color{blue}\ttfamily,
stringstyle=\color{red}\ttfamily,
commentstyle=\color{green}\ttfamily,
morecomment=[l][\color{magenta}]{\#},
columns=fullflexible }

\usepackage{titlesec}
\titleformat{\chapter}[hang]{\LARGE\bfseries}{\thechapter{.} }{0pt}{\LARGE\bfseries}
\titleformat*{\section}{\Large\bfseries}
\titleformat*{\subsection}{\large\bfseries}

\begin{document}
	
	\tableofcontents

	\chapter*{Введение}
	\addcontentsline{toc}{chapter}{Введение}
	
	\vspace{-0.5cm}\hspace{0.6cm}Многопоточность (как доктрину программирования) не следует путать ни с многозадачностью, ни с многопроцессорностью, несмотря на то, что операционные системы, реализующие многозадачность, как правило, реализуют и многопоточность. Смысл многопоточности - квазимногозадачность на уровне одного исполняемого процесса\cite{Volkova}\cite{thread}.
		
	\vspace{0.3cm}В данной работе требуется рассмотреть алгоритм Винограда для умножения матриц в однопоточной и многопоточной реализациях, а также провести сравнительный анализ.
	
		\vspace{0.3cm}Цель работы: изучение многопоточности и получение практики на примере алгоритма Винограда для перемножения матриц.
	
		\vspace{0.3cm}Задачи работы:
	\begin{enumerate}
		\item Разработка и реализация алгоритмов.
		\item Исследование временных затрат алгоритма.
		\item Описание и обоснование полученных результатов.
	\end{enumerate}
	

    \chapter{Аналитический раздел}
    
   	\vspace{-0.5cm}\hspace{0.6cm}В данном разделе будет описан алгоритм Винограда для перемножения матриц.
   	
   	
	\section{Выбор СУБД}
	
	\subsection{MongoDB}
	
	\hspace{0.7 cm}MongoDB это кросс - платформенная, документо-ориентированная база данных, которая обеспечаивает высокую производительность и лёгкую масштабируемость. В основе данной БД лежит  концепция коллекций и документов.
	
	База данных представленна в виде физического хранилища коллекций. Каждая БД имеет свой собственный набор файлов в файловой системе. Обычно, один MongoDB сервер имеет несколько БД \cite{mongodb}.
	
	\begin{lstlisting}[label=monogdb-document,caption=Пример документа в базе данных Mongo]
	{
		_id: ObjectId(7bf78ad8902c)
		title: 'MongoDB', 
		description: 'Simple MongoDB Database',
		by: 'proselyte',
		url: 'proselyte.net',
		tags: ['proselyte tutorials', 'NoSQL', 'MongoDB'], 
		developers: [	
		{
			developer:'developer1',
			specialty: 'Java Developer'
		},
		{
			developer: 'developer2'
			specialty: 'C++ Developer'
			}
		]
	}
	\end{lstlisting}
	
	
	\subsection{SQLite}
	
	\hspace{0.7 cm}SQLite - это библиотека языка Си, которая реализует небольшой, быстрый, автономный, высоконадежный, полнофункциональный компонент SQL database engine. SQLite - это самый распространенный движок баз данных в мире. SQLite встроен во все мобильные телефоны и большинство компьютеров и поставляется в комплекте с бесчисленными другими приложениями, которые люди используют каждый день.
	
	Формат файла SQLite является стабильным, кросс-платформенным и обратно совместимым \cite{sqlite}.
	
	\subsection{Coredata}
	
	\hspace{0.7 cm}Что-то \cite{coredata}
	
	\subsection{Realm}
	
	\subsection{Firebase realtime database}
	
	\subsection{Firebase firestore}
	
	Алгоритм Винограда это модифицированная версия классического алгоритма, где часть процессов высчитывается заранее. Эти вычисления позвовлят разбить вычисления алгоритма на потоки. Пусть есть две матрицы A и B размеров nxk, kxm соответственно. Тогда результатом перемножения этих двух матриц будет матрица C размера nxm \ref{for:lol1}.
	
	\begin{equation}
	A_{nk} * B_{km} = C = \begin{pmatrix}
		c_{11} & c_{12} & \cdots & c_{1m} \\
		c_{21} & c_{22} & \cdots & c_{2m} \\         
		\vdots & \vdots & \ddots & \vdots \\
		c_{n1} & c_{n2} & \cdots & c_{nm} \\
	\end{pmatrix}
	\label{for:lol1}
	\end{equation}
	
	Если посмотреть на результат умножения двух матриц, то можно заметить, что каждый элемент в нем представляет собой скалярное произведение соответствующих строки и столбца исходных матриц. Также, такое умножение позволяет сделать предварительную обработку заранее \cite{Volkova}
	
	Пусть два вектора $$V = (v1, v2, v3, v4), $$
	 
	$$W = (w1, w2, w3, w4)$$
	
	Тогда их скалярное произведение равно: 
	$$V * W = v1w1 + v2w2 + v3w3 + v4w4$$
	Это равенство можно переписать в виде: 
	$$V * W = (v1 + w2)(v2 + w1) + (v3 + w4)(v4 + w3) - 
	v1v2 - v3v4 - w1w2 - w3w4$$
	

	\section{Многопоточность}
	
	\hspace{0.6cm}Существуют зеленые и нативные потоки. Зеленые потоки - это потоки выполнения, управление которыми вместо операционной системы выполняет виртуальная машина. Программа написанная на языке, поддерживающим зеленые потоки, только эмитирует многопоточность. 
	
	\vspace{0.3cm}На многоядерных процессорах реализация нативных потоков может автоматически назначать работу нексольким процессорам, а реализация зеленых потоков не может назначить работу нескольким процессорам.
	
	\vspace{0.3cm}Поток выполнения - наименьшая единица обработки, исполнение которой может быть назначено ядром операционной системы. Реализация потоков выполнения и процессов в разных операционных системах отличается друг от друга, но в большинстве случаев поток выполнения находится внутри процесса. Несколько потоков выполнения могут существовать в рамках одного и того же процесса и совместно использовать ресурсы, такие как память, тогда как процессы не разделяют этих ресурсов\cite{bmstu}
	
	\section{Вывод}
	
	В данном разделе был описан алгоритм Винограда перемножени матриц.
	

	\chapter{Конструкторский раздел}
	
	\vspace{-0.6cm}\hspace{0.6cm} В данном разделе будет приведена блок-схема алгоритма Винограда для перемножения матриц, описано для каких частей алгоритма выделялись потоки, и каким образом это было реализовано.
	
	\section{Разработка алгоритмов}
	
	\hspace{0.6cm}В данном пункте представлена реализация алгоритма Винограда на рис. \ref{scheme} \cite{scheme}. 
	
	\newpage
	
%	\begin{figure}[ht!]
%		\label{scheme}
%		\centering
%		\includegraphics[scale=0.7]{scheme.png}
%		\caption{Схема алгоритма Винограда}
%	\end{figure}



	\newpage
	
	Для реализации многопоточной версии алгоритма можно выделить 4 основные части
	
	\begin{enumerate}
		\item Создание и инициализация MulH(A).
		\item Создание и инициализация MulV(B).
		\item Основные и дополнительные(для входных матриц нечетной размерностей) вычисления матрицы произведения(C).
	\end{enumerate}

	Части A, B разбиваются по потокам. Поток для C части не выделяется, пока не выполнятся A и B части.

	
	\section{Вывод}
	В данном разделе была приведена схема алгоритма, было описано каким образом выделялись потоки в реализованном алгоритме Винограда.
	
	\newpage
	
	\chapter{Технологический раздел}
	\vspace{-0.5cm}В данном разделе будут рассмотренны требования к разрабатываемому программному обеспечиванию, средства, использованные в процессе разработки для реализации поставленных задач, а также представлены листинги кода программы.
	
	\section{Требования к программному обеспечению}
	Программное обеспечивание должно реализовывать алгоритм Винограда для перемножения матриц в однопоточной и многопоточных реализациях.
	
	\section{Средства реализации}
	\hspace{0.6cm}Для выполнения поставленной задачи был использован язык программирования С++. Среда для разработки QtCreator. Данная среда разработки содержит в себя встроенную библиотеку для создания оконных приложений. Для измерения процессорного времени была взята функция rtdsc из библиотеки time.h. Для задержки времени для осуществления покадровой анимации была использована функция currentTime из встроенной библиотеки QTime.
	
	\vspace{0.2cm}Версия компилятора C++: GNU++11 [-std=gnu++11]
	
	
	\section{Листинг кода}
	\hspace{0.6cm}На основе схема, приведенной в конструкторском разделе, в соответствии с указанными требованиями к реализации с использованием языка С++ было разработано программное обеспечение, содержащее реализации выбранных алгоритмов. В данном пункте приведен листинги \ref{code1}-\ref{code2} реализации алгоритма\cite{prata}.

	\begin{lstlisting}[label=code1,caption=Реализация алгоритма Винограда]
	void Vinograd_n_thread(matrix_type &a, matrix_type &b, matrix_type &c, int n)
	{
	vector<int> row(a.n);
	vector<int> column(b.m);
	
	vector<thread> threads;
	
	unsigned int n1 = a.n / 2;
	zeroing(c.matrix, c.n, c.m);
	\end{lstlisting}

	\begin{lstlisting}[label=code2,caption=Основные вычисления для многопоточной реализации алгоритма Винограда]
	void create_mulH(int **&A, vector <int>& row, const unsigned int &M_start, const unsigned int &M_end, const unsigned int &N)
	{
	
	for (unsigned i = M_start; i < M_end; i++) {
	//cout << this_thread::get_id()<< endl;
	for (unsigned k = 0; k < N / 2; k++) {
	row[i] += A[i][2 * k] * A[i][2 * k + 1];
	}
	}
	}
	
	void create_mulV(int **&B, vector <int>& column, const unsigned int &Q_start, const unsigned int &Q_end, const unsigned int &N)
	{
	
	for (unsigned i = Q_start; i < Q_end; i++) {
	//cout << this_thread::get_id()<< endl;
	for (unsigned k = 0; k < N / 2; k++) {
	column[i] += B[2 * k][i] * B[2 * k + 1][i];
	}
	}
	}
	
	void calculate(int **&A, int **&B, int **&C, vector <int> &row, vector <int> &column, const unsigned int &M, const unsigned int &N, const unsigned int &Q)
	{
	for (unsigned i = 0; i < M; i++)
	for (unsigned j = 0; j < Q; j++) {
	if (N % 2 == 0)
	C[i][j] = -row[i] - column[j];
	else
	C[i][j] = -row[i] - column[j] + A[i][N - 1] * B[N - 1][j];
	
	for (unsigned k = 0; k < N / 2; k++) {
	C[i][j] = C[i][j] + (A[i][k << 1] + B[k << 1 | 1][j]) *
	(A[i][k << 1 | 1] + B[k << 1][j]);
	}
	}
	
	}
	
	void calculate1(matrix_type &a, matrix_type &b, matrix_type &c, vector <int> &row, vector <int> &column, const unsigned int n_start, unsigned int n_end)
	{
	int sum = 0;
	
	for (unsigned i = n_start; i < n_end; i++) {
	//cout << this_thread::get_id()<< endl;
	for (unsigned j = 0; j < b.m; j++) {
	
	sum = -row[i] - column[j];
	
	for (unsigned k = 0; k < a.m / 2; k++) {
	sum += (a.matrix[i][2*k] + b.matrix[2*k+1][j]) *
	(a.matrix[i][2*k+1] + b.matrix[2*k][j]);
	}
	
	if (a.m % 2 == 1)
	sum += a.matrix[i][a.m - 1] * b.matrix[b.n - 1][j];
	
	c.matrix[i][j] = sum;
	}
	}
	
	}
	
	\end{lstlisting}

	\newpage

	\section{Вывод}
	В данном разделе были рассмотрены требования к разрабатываемому программному обеспечению, средства, использованные в процессе разработки, а также был представлены листинги кода реализации алгоритма Винограда.

			
	\chapter{Исследовательский раздел}
	
	\vspace{-0.6cm}\hspace{0.5cm}В данном разделе будет приведено экспериментальное исследование временных затрат разработанного программного обеспечения, вместе в подробным сравнительным анализом реализованных алгоритмов на основе экспериментальных данных.
	
	\section{Сравнительный анализ}
	
	\hspace{0.6cm}Замеры времени выполнялись на квадратных матрицах размеров от 100х100 до 1000х1000 с интервалом в 100 элементов. Также замеры времени проводились над квадратными матрицами нечетной размерностью  от 101х101 до 1001х1001 с шагом 100. В табл. \ref{table1}-\ref{table2} и рис. \ref{pic1}-\ref{pic2} представлены результаты замеров в  секундах $ * 10^{-2}$.
	
	
	
	\begin{table}[ht!]
		\caption{Сравнение времени работы однопоточной и многопоточной версий алгоритма в единицах измерения библиотеки chrono}
		\label{table1}
		\begin{center}
			\begin{tabular}{|c|c|c|c|c|c|c|}
				\hline
				\bf{Размерность матриц} & \bf{Поток 1} & \bf{Поток 2} & \bf{Поток 3} & \bf{Поток 4} &\bf{Поток 5} & \bf{Поток 6} \\\hline
				
				$100$ & $4$ & $2$ & $1$ & $1$ & $1$ & $1$ \\ \hline
				
				$200$ & $36$ & $17$ & $7$ & $6$ & $6$ & $6$\\\hline
				
				$300$ & $128$ & $56$ & $36$ & $35$ & $27$ & $27$\\\hline
				
				$400$ & $334$ & $165$ & $111$ & $89$ & $90$ & $77$\\\hline
				
				$500$ & $590$ & $331$ & $205$ & $176$ & $161$ & $148$\\\hline
				
				$600$ & $932$ & $535$ & $369$ & $279$ & $250$ & $241$\\\hline
				
				$700$ & $1504$ & $857$ & $591$ & $505$ & $465$ & $423$\\\hline
				
				$800$ & $2329$ & $1261$ & $815$ & $710$ & $731$ & $675$\\\hline
							
			    $900$ & $3808$ & $2818$ & $1570$ & $1166$ & $1131$ & $1084$ \\\hline
				
				$1000$ & $10691$ & $4392$ & $2743$ & $2169$ & $1816$ & $1529$\\\hline
			\end{tabular}
		\end{center}
	\end{table}



	\newpage

\begin{table}[ht!]
	\caption{Сравнение времени работы однопоточной и многопоточной версий алгоритма в единицах измерения библиотеки chrono}
	\label{table2}
	\begin{center}
		\begin{tabular}{|c|c|c|c|c|c|c|}
			\hline
			\bf{Размерность матриц} & \bf{Поток 1} & \bf{Поток 2} & \bf{Поток 3} & \bf{Поток 4} & \bf{Поток 5} & \bf{Поток 6} \\\hline
			
			$101$ & $4$ & $2$ & $1$ & $1$ & $1$ & $1$\\\hline
			
			$201$ & $24$ & $15$ & $7$ & $11$ & $7$ & $7$\\\hline
			
			$301$ & $116$ & $62$ & $37$ & $26$ & $31$ & $30$\\\hline
			
			$401$ & $320$ & $153$ & $111$ & $86$ & $84$ & $75$\\\hline
			
			$501$ & $531$ & $306$ & $209$ & $163$ & $150$ & $142$\\\hline
			
			$601$ & $993$ & $492$ & $333$ & $260$ & $249$ & $242$\\\hline
			
			$701$ & $1487$ & $853$ & $556$ & $455$ & $556$ & $428$\\\hline
			
			$801$ & $2482$ & $1154$ & $850$ & $699$ & $729$ & $643$\\\hline
			
			$901$ & $6480$ & $2810$ & $1562$ & $1093$ & $1175$ & $1362$ \\\hline
			
			$1001$ & $8949$ & $3892$ & $2645$ & $2179$ & $2260$ & $2054$\\\hline
		\end{tabular}
	\end{center}
\end{table}

	%\newpage
	
	\begin{table}[ht!]
		\caption{Время работы программы для 7 и 8, и 100 потоков}
		\label{table3}
		\begin{center}
			\begin{tabular}{|c|c|c|c|}
				\hline
				\bf{Размерность матриц} & \bf{Поток 7} & \bf{Поток 8} & \bf{Поток 100}\\\hline
				
				$100$ & $1$ & $1$ & $2$\\\hline
				
				$200$ & $7$ & $6$ & $6$\\\hline
				
				$300$ & $26$ & $26$ & $24$\\\hline
				
				$400$ & $70$ & $66$ & $67$\\\hline
				
				$500$ & $132$ & $126$ & $132$\\\hline
				
				$600$ & $218$ & $210$ & $236$\\\hline
				
				$700$ & $410$ & $370$ & $360$\\\hline
				
				$800$ & $595$ & $581$ & $541$\\\hline
				
				$900$ & $1133$ & $937$ & $879$\\\hline
				
				$1000$ & $1356$ & $1482$ & $1410$\\\hline
			\end{tabular}
		\end{center}
	\end{table}

	Из данных графиков видно, что присутствие хотя бы двух потоков в несколько раз эффективнее, в сравнению с однопоточной реализацией алгоритма. Разница по времени выполнения в многопоточной реализации алгоритма не зависит от четной или нечетной размерности матриц. 
	\newpage
	
	\section{Вывод}
	
	\hspace{0.6cm}В данном разделе было проведено исследование однопоточной и многопоточных версий алгоритма. Приведены графики зависимостей времени работы алгоритма от размерности матриц.
	
	\vspace{0.3cm}Среди всех версий алгоритма самыми лучшим оказались версии, где задействовалось не менее четырех потоков. Многопоточная версия алгоритмов показала хороший реазультат относительно однопоточной версии: двухпоточная реализация работает быстрее в 2.5 раза по сравнению с однопоточной реализацией при размерности матриц равной 1000.

	\newpage

	\chapter*{Заключение}
	\addcontentsline{toc}{chapter}{Заключение}
		\vspace{-0.6cm}\hspace{0.5cm}В ходе выполнения данной лабораторной работы были изучены и реализованы различные алгоритмы перемножения матриц. В аналитической части было приведено описание алгоритмов. В конструкторской части были представлены блок-схемы алгоритмов. Также был выполнен расчет сложности алгоритмов. В экспериментальной части проведен сравнительный анализ временных затрат, после которого была выявлена ощутимая эффективность многопоточного программирования по отношению к стандартному, однопоточному. 
		
		\vspace{0.3cm}Для перемножения двух квадратных матриц размерностью 1000х1000 результаты оказались следующими:
	\begin{itemize} 
		\item Однопоточная реализация алгоритма проигрывает по времени работы двухпоточной в ~2.4 раза.
		\item Однопоточная реализация алгоритма проигрывает по времени работы четырехпоточной в ~4 раз.
		\item Однопоточная реализация алгоритма проигрывает по времени работы шестипоточной в ~7 раз.
		\item Однопоточная реализация алгоритма проигрывает по времени работы восьмипоточной в ~7.2 раз.
		\item Однопоточная реализация алгоритма проигрывает  по времени работы стопоточной в ~7.58 раз.
	\end{itemize}
	
	\newpage

\begin{thebibliography}{3}
	\bibitem{mongodb}
	MongoDb [Электронный ресурс]. - Режим доступа: https://www.mongodb.com, свободный. (Дата обращения: 02.06.2020 г.)
	\bibitem{sqlite} 
	SQLite [Электронный ресурс]. - Режим доступа: https://www.sqlite.org/ свободный. (Дата обращения: 02.06.2020 г.)
	\bibitem{coredata} 
	Coredata [Электронный ресурс]. - Режим доступа: https://developer.apple.com/documentation/coredata свободный. (Дата обращения: 02.06.2020 г.)
	\bibitem{prata}
	Стивен Прата "Язык программирования С++." (2012 г.) [Письменный ресурс] - ISBN 5-94836-005-9
	\bibitem{scheme}
	Построение блок-схем [Электронный ресурс]. - Режим доступа https://pro-prof.com/archives/1462, свободный. (Дата обращения: 29.10.2019 г.)
	\bibitem{thread}
	Многопоточность на корабликах [Электронный ресурс]. - Режим доступа https://habr.com/ru/post/352374/, свободный. (Дата обращения: 19.11.2019 г.)
\end{thebibliography}
\end{document}
